\begin{appendices}

\chapter{Description du sujet}
\label{appendix:description}

\begin{figure}[h]
  \includegraphics[width=1\linewidth]{content/imgs/description.png}
  \caption*{Description du projet proposé par ma superviseure Dr. Noreen Izza Arshad}
\end{figure}


\begin{landscape}
\chapter{Stutter Manager v3}
\label{appendix:old_app}

\begin{figure}[h]
  \centering
  \begin{subfigure}{.25\textwidth}
    \centering
    \includegraphics[width=.75\linewidth]{content/imgs/old_app_1.jpg}
    \caption{Page principale (exercices)}
  \end{subfigure}%
  \begin{subfigure}{.25\textwidth}
    \centering
    \includegraphics[width=.75\linewidth]{content/imgs/old_app_2.jpg}
    \caption{Informations générales}
  \end{subfigure}%
  \begin{subfigure}{.25\textwidth}
    \centering
    \includegraphics[width=.75\linewidth]{content/imgs/old_app_3.jpg}
    \caption{Exercice : Metronome}
  \end{subfigure}%
  \begin{subfigure}{.25\textwidth}
    \centering
    \includegraphics[width=.75\linewidth]{content/imgs/old_app_4.jpg}
    \caption{Progression de l'exercice metronome}
  \end{subfigure}
  \caption*{Captures d'écran de Stutter Manager v3}
\end{figure}

\end{landscape}


\chapter{Étude comparative}
\label{appendix:market}
\begin{figure}[h]
  \includegraphics[width=1\linewidth]{content/imgs/market.png}
  \caption*{Étude comparative des applications disponibles sur le Play Store en comparaison avec les fonctionnalités prévues pour Stuttherapy}
\end{figure}

\chapter{Software requirements specification}
\label{appendix:srs}
\begin{figure}[h]
  \includegraphics[width=0.7\linewidth]{content/imgs/srs_contents.png}
  \caption*{Table des matières du \textit{Software requirements specification}}
\end{figure}

\begin{displayquote}
The software requirements specification lays out functional and non-functional requirements, and it may include a set of use cases that describe user interactions that the software must provide to the user for perfect interaction.
\end{displayquote}
\hspace*{\fill} \textit{Wikipedia - Software Requirements Specification}


\begin{displayquote}
La spécification des exigences logicielles définit les exigences fonctionnelles et non fonctionnelles. Elle peut inclure un ensemble de cas d'utilisation décrivant les interactions de l'utilisateur que le logiciel doit fournir à l'utilisateur pour obtenir une interaction parfaite.
\end{displayquote}
\hspace*{\fill} \textit{Traduction du passage ci-dessus}


\chapter{Diagramme IHM}

\chapter{Maquettes}

\chapter{Diagramme de classe}

\chapter{Stuttherapy - Captures d'écran}


\end{appendices}
