\section{Choix des technologies}

\subsection{Création de l'application}

\textit{Android} et \textit{iOS} sont les deux systèmes d'exploitation mobile les plus utilisés sur les smartphones en 2018, ils se partagent respectivement $74.45\%$ et $22.5\%$ du marché \cite{market_share}. Il existe principalement deux solutions pour developper une application mobile sur ces plateformes :

\begin{itemize}
  \item \textit{Application native} : en utilisant les SDK (\textit{kit de developpement logiciel}) spécifiques à chaque plateforme (\textit{iOS} et \textit{Android}) qui fournissent des outils de developpement et de debuggages permettant de créer des applications pour ces plateformes ;
  \item \textit{Application hybride} : en utilisant un \textit{framework multi-plateforme} permettant la création d'applications sur plusieurs plateformes avec le même code (sans utiliser directement les SDK de ces dernières).
\end{itemize}

Les avantages des applications natives comparé aux applications hybrides sont :

\begin{itemize}
  \item La disponibilité de toutes les, et notamment des dernières, fonctionnalités des différents systèmes d'exploitation ;
  \item La performance de l'application en accédant directement au système d'exploitation de la plateforme sur laquelle l'application est developpé.
\end{itemize}

L'inconvénient majeur est qu'il est necessaire de developper, maintenir et deployer la même application 2 fois (une fois pour \textit{Android} et une fois pour \textit{iOS}).

Les framework mutli-plateforme permettent d'écrire des application pour \textit{iOS} et \textit{Android} avec le même code (bien qu'il est aussi possible d'écrire du code spécifique à une plateforme) permettant d'avoir un \textit{time to market} plus court (temps nécessaire pour rendre le produit disponible sur le marché). Ces frameworks proposent aussi souvent des solutions modernes permettant encore de réduire ce \textit{time to market} en simplificant le developpement de l'application. Les inconvénients de ces frameworks sont propres à chacun. Pour commencer la comparaison j'ai choisi les frameworks activement developpés avec une communauté importante et active. Les frameworks les plus populaires en 2019 sont \textbf{Flutter} (71 662 stars avec 422 contributeurs sur Github \cite{flutter}), \textbf{React Native} (79 513 stars avec 1990 cotnributeurs \cite{react}) et \textbf{Ionic} (38 705 stars avec 331 contributeurs \cite{ionic}).

\begin{wrapfigure}{r}{0.25\textwidth}
  \includegraphics[width=100pt]{content/imgs/flutter.png}
\end{wrapfigure}

J'ai finalement opté pour \textbf{Flutter}, le framewok multi-plateforme developpé par \textit{Google}. Flutter utilise \textit{Dart}, un langage de programmation orienté objet utilisant un \textit{garbage collector} à la manière de \textit{Java}. J'ai choisi Flutter pour plusieurs raisons :

\paragraph{Performance}
Le code de l'application est compilé à l'avance en code ARM natif et non pas au moment de l'exécution comme React Native le fait, ce qui permet d'avoir des performances similaires aux applications natives. Ça semble être le framework de developpement mobile le plus performance du marché.

\paragraph{Composents graphiques}
Flutter gère entièrement le rendu des éléments graphiques sur son propre canvas (\textit{= zone où sont dessiné les éléments graphique de l'application}) sans utiliser les composants graphiques natifs aux plateformes (comme le fait React Native). On peux donc gérer au pixels près les éléments graphiques utilisés. L'application aura la même apparence indépendemment des versions des systèmes d'exploitation. Cela permet de gérer précisement ce qu'il sera affiché sur tous les appareils.


\subsection{Stockage de données dans le cloud}

L'application doit proposer le partage de données entres les bègues et les thérapistes. Pour ce faire les données doivent être stockés dans le cloud. J'ai choisis d'utiliser un système d'authentification pour les utilisateurs avec adresse mail et mot de passe pour qu'ils puissent gérer leurs données et que ces dernière soient uniquement accéssible aux personnes autorisés (et non pas sur un espace "publique").

\begin{wrapfigure}{r}{0.25\textwidth}
  \includegraphics[width=100pt]{content/imgs/firebase.png}
\end{wrapfigure}
J'ai choisis d'utiliser \textbf{Firebase}, en particuler la base de données noSQL \textbf{Firestore} et le système d'authentification de Firebase pour gérer les utilisateurs. J'ai choisi Firebase pour plusieurs raisons :

\paragraph{Facilité d'utilisation}
Google developpe et maintient à jour un ensemble de plugins permettant d'utiliser les produits Firebase facilement et rapidement :  \textit{FlutterFire} \cite{flutterfire}. Les plugins offre les dernière fonctionnalités des produits Firebase.

\paragraph{Evolution de l'application sur d'autres plateformes}
Firebase est largement utilisé aujourd'hui, les produits Firebase peuvent être utilisé aussi bien sur les web que sur applications mobiles facilement grâce à un grand nombre de plugins developpé par les communautés des différents framework / langages. Il peux aussi être utilisé directement en utilisant les API des produits avec les langages supportés.

\paragraph{Evolution de l'utilisation du Cloud au sein de l'application}
Pour l'instant, seuls \textit{Firestore} et de le système d'authentification de Fireabase sont utilisés. Cependant Firebase intègre bien d'autre service pouvant potentiellement être utilisés par l'application. Parmis eux, on peux cité \textit{Firebase Analytics} permettant d'analyser le comportement des utilisateurs sur l'application, \textit{Firebase AdMob} pour intégrer des publicités au sein de l'application ou encore \textit{Cloud Functions} pour exécuter des fonctions sur des serveurs. Tous ces produits peuvent être étroitement utilisé ensemble, par exemple \textit{Cloud Firestore} peut automatiquement déclencher une fonction de  \textit{Cloud Functions} lors de l'ajout d'un nouveau document dans la base de données. Donc, \textbf{sans} modifier le code de l'application côté client, il est possible d'enrichir ou de modifier les fonctionnalités de l'application en utilisant les différents services Firebase (authentification, publicité, base de données, stockage, etc.).
