\section{Choix des technologies}

\subsection{Création de l'application}

\textit{Android} et \textit{iOS} sont les deux systèmes d'exploitation mobile les plus utilisés sur les smartphones en 2018, ils se partagent respectivement $74.45\%$ et $22.5\%$ du marché \cite{market_share}. Il existe principalement deux solutions pour développer une application mobile sur ces plateformes :

\begin{itemize}
  \item \textbf{Application native} : en utilisant le \textit{SDK} (\textit{kit de developpement logiciel}) spécifique à chaque plateforme (\textit{iOS} ou \textit{Android}) qui fournit les outils de développement et de debogages permettant de créer des applications sur la plateforme concernée ;
  \item \textbf{Application hybride} : en utilisant un \textit{framework multi-plateforme} permettant la création d'applications sur plusieurs plateformes avec le même code (sans utiliser directement les \textit{SDK} de ces dernières).
\end{itemize}

Les avantages des applications natives comparé aux applications hybrides sont :

\begin{itemize}
  \item La disponibilité de toutes les fonctionnalités des différents systèmes d'exploitation, notamment des dernières ;
  \item La performance de l'application en accédant directement au système d'exploitation de la plateforme sur laquelle l'application est développée.
\end{itemize}

L'inconvénient majeur est qu'il est nécessaire de développer, maintenir et déployer la même application deux fois (une fois pour \textit{Android} et une fois pour \textit{iOS}).

Les frameworks multi-plateformes permettent d'écrire des applications pour \textit{iOS} et \textit{Android} avec le même code (bien qu'il soit aussi possible d'écrire du code spécifique à une plateforme) permettant d'avoir un \textit{time to market} réduit (temps nécessaire pour rendre le produit disponible sur le marché). Ces frameworks proposent aussi souvent des solutions modernes permettant encore de réduire ce \textit{time to market} en simplifiant le développement de l'application. Les inconvénients de ces frameworks sont propres à chacun. Pour commencer la comparaison, j'ai choisi les frameworks activement développés avec une communauté importante et active. Les frameworks les plus populaires en 2019 sont \textbf{Flutter} (71 662 stars avec 422 contributeurs sur \textit{Github} \cite{flutter}), \textbf{React Native} (79 513 stars avec 1990 cotnributeurs \cite{react}) et \textbf{Ionic} (38 705 stars avec 331 contributeurs \cite{ionic}).

\begin{wrapfigure}{r}{0.25\textwidth}
  \includegraphics[width=100pt]{content/imgs/flutter.png}
\end{wrapfigure}

J'ai finalement opté pour \textbf{Flutter}, le framework multi-plateforme développé par \textit{Google}. \textit{Flutter} utilise \textit{Dart}, un langage de programmation orienté objet utilisant un \textit{garbage collector}, comme \textit{Java} par exemple. J'ai choisi \textit{Flutter} pour plusieurs raisons :

\paragraph{Performance}
Le code de l'application est compilé à l'avance en code \textit{ARM} natif (\textit{AOT-compiled}) et non pas au moment de l'exécution comme \textit{React Native} le fait, ce qui permet d'avoir des performances similaires aux applications natives. Ça semble être le framework de développement mobile le plus performance du marché.

\paragraph{Composents graphiques}
\textit{Flutter} gère entièrement le rendu des éléments graphiques sur son propre canvas (\textit{zone où sont dessinés les éléments graphiques de l'application}) sans utiliser les composants graphiques natifs de la plateforme sur laquelle elle est compilée et installée (comme le fait \textit{React Native}). Il est donc possible de gérer au pixel près les éléments graphiques utilisés. L'application aura la même apparence indépendamment des versions des systèmes d'exploitation sur lesquels l'application est installée. Cela permet de gérer précisément ce qu'il sera affiché sur tous les appareils sans tester l'application sur une multitude de téléphones.


\subsection{Stockage des données dans le cloud}

L'application doit proposer le partage de données entre les bègues et les orthophonistes. Pour ce faire, les données doivent être stockées dans le cloud. Pour que les utilisateurs puissent gérer leurs données et que ces dernières soient uniquement accessibles aux personnes autorisées (et non pas sur un espace de stockage public), ils devront posséder un compte utilisateur (adresse mail / mot de passe).

\begin{wrapfigure}{r}{0.25\textwidth}
  \includegraphics[width=100pt]{content/imgs/firebase.png}
\end{wrapfigure}
J'ai choisi d'utiliser les services \textbf{Firebase}, en particulier la base de données noSQL \textbf{Firestore} pour stocker les exercices et le système d'authentification de Firebase pour gérer les utilisateurs. J'ai choisi \textit{Firebase} pour plusieurs raisons :

\paragraph{Facilité d'utilisation}
\textit{Google} développe et maintient à jour un ensemble de plugins permettant d'utiliser les produits \textit{Firebase} facilement et rapidement sur les applications mobiles \textit{Flutter} :  \textit{FlutterFire} \cite{flutterfire}. Ces plugins offrent les dernières fonctionnalités des services proposés par \textit{Firebase}.

\paragraph{Évolution de l'application sur d'autres plateformes}
\textit{Firebase} est largement utilisé dans le monde informatique aujourd'hui, les services de \textit{Firebase} peuvent être utilisés n'importe où facilement grâce à un grand nombre de plugins développés par \textit{Google} et par les communautés des différents frameworks / langages ou en utilisant directement en utilisant les API des services.

\paragraph{Évolution de l'utilisation du cloud au sein de l'application}
Pour l'instant, seuls les services \textit{Firestore}, \textit{Firebase Authentication} et \textit{Firebase Storage} sont prévus d'être utilisés pour l'application. Cependant, \textit{Firebase} intègre bien d'autres services pouvant potentiellement être utile pour l'application. Parmi eux, on peut citer \textit{Firebase Analytics} permettant d'analyser le comportement des utilisateurs sur l'application, \textit{Firebase AdMob} pour intégrer des publicités au sein de l'application ou encore \textit{Cloud Functions} pour exécuter des fonctions sur des serveurs. Tous ces produits peuvent être étroitement utilisés en harmonie, par exemple \textit{Cloud Firestore} peut automatiquement déclencher une fonction de \textit{Cloud Functions} lors de l'ajout d'un nouveau document dans la base de données. Donc, \textbf{sans} modifier le code de l'application côté client, il est possible d'enrichir ou de modifier les fonctionnalités de l'application en utilisant les différents services \textit{Firebase} (authentification, publicité, base de données, stockage, etc.).
