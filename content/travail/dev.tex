\section{Developpement de l'application, tâches à finir, futur de l'application}

\subsection{Fonctionnalités developpées}
L'application se nomme \textbf{Stuttherapy}. Voici la liste (presque) exhaustive des fonctionnalités developpées durant les 7 semaines consacrées au developpement :

\begin{itemize}

  \item \textbf{Pour les utilisateurs bègues :}
  \begin{itemize}
    \item Trois différents exercices disponibles : \textit{metronome}, \textit{mirroring} et \textit{reading} (voir \nameref{sec:resume_cdc} pour plus d'informations concernant le but de ces exercices) ;
    \item Progression pour chacun des exercices disponible sous deux formes :
    \begin{itemize}
      \item Liste de tous les exercices effectués avec les informations et données de l'exercice en question : date de création, éventuel enregistrement audio ou vidéo et les éventuls mots dont l'utilisateur n'a pas réussi à prononcer correctement ;
      \item Un représentation graphique \textit{ligne} du pourcentage de réussite de prononciation des mots au cours des différents exercices. L'utilisateur peut changer la fenêtre de temps du graphique pour afficher sa progression hebdomadaire, mensuelle ou annuelle.
    \end{itemize}
    \item Accès à sa liste de mots \textit{enregistrés}. L'utilisateur peut, si il le souhaite, indiqué les mots avec lesquels il a eu des difficultés lors des exercices, ils seront alors automatiquement enregistré dans cette liste.
    \item Synchroniser et récupérer des exercices dans le cloud : l'utilisateur peut choisir de synchroniser ou desynchroniser les exercices qu'il veut dans le cloud grâce à un compte utilisateur. Il peut récupérer les exercices synchronisé dans le cloud sur son appareil (s'il change de téléphone par exemple).
    \item Ajouter un orthopédistes autorisé à avoir accès à ces exercices synchronisés dans le cloud
  \end{itemize}

  \item \textbf{Pour les orthopédistes :}
  \begin{itemize}
    \item figure
    \item d
  \end{itemize}
\end{itemize}

Des captures d'écran de l'application sont disponibles dans l'annexe \ref{appendix:screenshots}.

\subsection{Reste à faire}


\subsection{Et ensuite ?}
