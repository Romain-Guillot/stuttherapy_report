% OK!
\section{Modélisation de l'interface graphique}

\subsection{\textit{Flowchart}}

À partir du cahier des charges de l'application, j'ai construit le \textit{flowchart} (représentation schématique d'un enchaînement d'actions) des interactions possibles de l'utilisateur sur les différentes pages de l'application, voir l'annexe \ref{appendix:ihm}. J'ai tout d'abord listé les différentes pages que doit composer l'application. Une page est représentée par un rectangle avec le nom de la page accompagné d'un bref descriptif. La navigation entre deux pages est représentée par une flèche directionnelle vers la page de destination avec le déclencheur qui provoque le changement de page (représenté par un losange).

\begin{figure}[H]
  \includegraphics[width=.8\linewidth]{content/imgs/ihm_ex.png}
  \caption{Exemple du flowchart entre les pages \textit{Exercice Homepage} et \textit{Therapy Feed}}
  \label{fig:flowchart}
\end{figure}


\subsection{\textit{Wireframes}}

Une fois, les différentes pages de l'application définies, je me suis attelé à élaborer les \textit{wireframes} de l'application. Un \textit{wireframe} est une maquette fonctionnelle, c'est-à-dire une représentation schématique de l'interface dans le but de définir les différents composants de l'interface sans se soucier des règles de design de celle-ci -- comme les couleurs principales de l'application ou les règles typographiques par exemple. Les \textit{wireframes} se concentrent sur l'ergonomie de l'application et non sur le desgin de celle-ci. Le \textit{wireframe} de la page principale d'un utilisateur bègue est disponible dans la figure \ref{fig:wireframe}.


\begin{figure}[H]
  \includegraphics[width=.3\linewidth]{content/imgs/wireframe_ex.png}
  \caption{\textit{Wireframe} de la page d'accueil d'un utilisateur bègue}
  \label{fig:wireframe}
\end{figure}
