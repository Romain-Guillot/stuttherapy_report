
\clearpage \ifodd\value{page}\hbox{\thispagestyle{empty}}\newpage\fi
\thispagestyle{empty}

\noindent\textbf{Étudiant :} Romain GUILLOT

\noindent\rule{\textwidth}{1pt}

\noindent\textbf{Année scolaire :} 2018 - 2019

\noindent\rule{\textwidth}{1pt}

\noindent\textbf{Entreprise :} Universiti Technologi PETRONAS \\
\noindent\textbf{Adresse postale :} Persiaran UTP, 32610 Seri Iskandar, Perak, Malaysia \\
\noindent\textbf{Téléphone :} +60 1-300-22 8887

\noindent\rule{\textwidth}{1pt}

\noindent\textbf{Tuteur de stage :} Noreen Izza Arshad \\
\noindent\textbf{Téléphone :} +60 5-368 7498\\
\noindent\textbf{Courriel :} noreenizza@utp.edu.my

\noindent\rule{\textwidth}{1pt}

\noindent\textbf{Enseignant-référent :} Jean-François Monin \\
\noindent\textbf{Téléphone :} +33 4 57 42 22 31 \\
\noindent\textbf{Courriel :} jean-francois.monin@univ-grenoble-alpes.fr

\noindent\rule{\textwidth}{1pt}

\noindent\textbf{Titre :} Développer une application mobile pour les thérapies électroniques pour les bègues

\noindent\rule{\textwidth}{1pt}

\noindent\textbf{Résumé :}
Dans le cadre de ma 4\up{ème} année d'études en informatique à Polytech Grenoble, j'ai effectué mon stage de fin d'année à \textit{Universiti Teknologi Petronas} en Malaisie. Durant ce stage de 12 semaines, j'ai été amené à concevoir et à développer une application mobile.

Le but de ce stage était de réaliser une application mobile visant à aider les personnes souffrant de bégaiement à corriger ce trouble de la parole. L'application est aussi destinée aux orthophonistes souhaitant suivre l'évolution de la progression de leurs patients utilisant l'application.

Ce document donne une vision globale du projet et des solutions apportées durant ce stage.
