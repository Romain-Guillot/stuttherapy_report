\chapter{Bilan et conclusion}
\label{chapter:bilan}

Ce stage a été très enrichissant dans le cadre de mes études d'ingénieurs. Le projet effectué durant ces 12 semaines étaient très complet, il était à l'état d'idée lors de mon arrivé, il est maintenant prêt à être mis en production. J'ai consolidé et découvert de nouvelles connaissances assez variées.

Premièrement, j'ai mis en pratique mes connaisances acquisent lors du cours de génie logiciel par \textbf{Didier Donzez} pour écrire les spécifications complètes de l'application après avoir analysé la marché dans lequel elle est destinée.

J'ai ensuite choisie les technologies les plus adaptées à ce projet au vu des spécifications du projet, du contexte dans lequel il s'insère et des coûts qu'elles engendrent. Puis, j'ai imaginé la conception d'un point de vue technique du projet pour définir les différents compostants du système leurs architectures individuel et au sein du projet. Après cette phase globale de conception du projet, je suis passé à la programmation de l'application. L'architecture technique du projet et le développement concret de celui-ci était les étapes avec lesquels j'étais le plus familier, notamment grâce à mes projets personnels ainsi qu'aux cours et aux conseils de \textbf{Oliver Gruber} sur la programmation object, les systèmes distribué et le projet de fin de 3\up{ème} année.

En dernier lieu, j'ai finalisé les derniers petits détails concernant l'application pour le deploiement de celle-ci sur le \textit{Play Store}. C'est notament sur cette partie de la réalisation de l'application que j'ai le plus de regrets. En effet, j'ai laissé cette étape pour la fin du projet alors qu'il aurait été préférable de mettre en place des outils d'intégration continu directement dès la première semaines afin d'améliorer le \textit{worflow} de mise en production au cours du projet. De plus, j'ai pas eu le temps à la fin du projet de m'attarder sur les solutions qui s'offrait à moi pour mettre en place ces outils d'intégration continue, j'aurais bien aimé developpé ce domaine compétences, en ayant très peu fait au cours de ces 2 années à Polytech et durant mes projets personnels.

Indépendament de ma volonté cette fois-ci, j'ai été seul à travailler sur ce projet et j'ai eu très peu de contraintes de la part de ma superviseure (d'un point de vue techniques ou manageriale par exemple). J'aurais grandement aimer être intégré au sein d'une équipe pour découvrir le travail en équipe d'un point de vue professionnel, avoir des contraintes et surtout avoir pouvoir discuter avec ingénieurs ou developpeur des différentes solutions d'implémentation des composants de l'application. L'environnement de travail sera donc un point crucial dans ma recherche de stage de 5\up{ème} année.
