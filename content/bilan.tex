% OK!
\chapter{Bilan et conclusion}
\label{chapter:bilan}

Ce stage fut une expérience très enrichissant au sein de mon cursus d'études d'ingénieur. Le projet effectué durant ces 12 semaines était très complet, il était à l'état d'idée lors de mon arrivée, il est maintenant prêt à être mis en production. J'ai consolidé et découvert de nouvelles connaissances assez variées.

Premièrement, j'ai mis en pratique mes connaissances acquises lors du cours de génie logiciel enseigné par \textbf{Didier Donzez} pour écrire les spécifications complètes de l'application après avoir analysé le marché dans lequel elle d'insert.

J'ai ensuite choisi les technologies les plus adaptées à ce projet au vu des spécifications du projet, du contexte dans lequel il s'insère et des coûts que ces technologies engendrent. Puis, j'ai imaginé la conception d'un point de vue technique du projet afin de définir les différents composants du système, leurs architectures au niveau individuel et au sein du projet. Après cette phase globale de conception du projet, je suis passé à la programmation de l'application. L'architecture technique du projet et le développement concret de celui-ci étaient les étapes avec lesquels j'étais le plus familier, notamment grâce à mes projets personnels ainsi qu'aux cours et aux conseils de \textbf{Oliver Gruber} sur la programmation objet, les systèmes distribués et le projet de fin de 3\up{ème} année.

En dernier lieu, j'ai finalisé les derniers détails concernant l'application pour le déploiement de celle-ci sur le \textit{Play Store}. C'est notamment sur cette partie de la réalisation de l'application que j'ai le plus de regrets. En effet, j'ai laissé cette étape pour la fin du projet alors qu'il aurait été préférable de mettre en place des outils d'intégration continue directement dès la première semaine afin d'améliorer le \textit{workflow} de mise en production tout au long du projet. De plus, je n'ai pas eu le temps à la fin du projet de m'attarder sur les solutions qui s'offraient à moi pour mettre en place ces outils d'intégration continue. Ayant très mis en pratique l'intégration continue au cours de ces deux années à Polytech, ni durant mes projets personnels, j'aurais bien aimé développer ce domaine de compétences.

Indépendamment de ma volonté, cette fois-ci, j'ai été seul à travailler sur ce projet et j'ai eu très peu de contraintes de la part de ma tutrice (d'un point de vue technique ou managériale par exemple). J'aurais grandement aimé être intégré au sein d'une équipe afin de découvrir le travail en équipe d'un point de vue professionnel, pour avoir des contraintes et surtout pour pouvoir discuter avec professionnels expérimentés des différentes solutions d'implémentation des différents composants de l'application. L'environnement de travail sera donc un point crucial dans ma recherche de stage de 5\up{ème} année.
